\chapter{Trabalhos Relacionados}
A seção a seguir tem por objetivo apontar alguns dos trabalhos já publicados e que tenham relação com o assunto desta pesquisa. 

\section{REDES NEURAIS NA PREVISÃO DO MERCADOS FINANCEIROS}
No trabalho de \citeonline[]{BOSAIPO} ele faz testes com vários tipos de redes neurais diferentes, a que atinge melhores resultados, segundo ele, é a rede com o algoritmo de aprendizado backpropagation randomized, esta rede conta com três camadas sendo 20 neurônios na primeira 6 na segunda e 2 na última. A função de ativação usada por ele nesta rede foi a hiperbólica entre a primeira e segunda camada, e a função de ativação linear entre a segunda e a última.

Embora a rede com as configurações citadas acima tenha tido um bom resultado, há um alto índice de erros o que torna a aplicação pouco confiável. Provavelmente, como o próprio autor sugere, os erros que a rede produziu foi por causa do curto período de tempo usado para testes, o autor usou como entrada apenas as cotações e os volumes negociados nos últimos 10 dias.

Seguindo a mesma linha de raciocínio de Bosaipo \citeonline[]{OCONNOR} implementam uma rede neural que tem como dados de entrada o valor de abertura e com esses dados tenta prever se a ação terá alta ou queda ao final do dia. Esta implementação é muito boa porém não aponta qual será o valor máximo que a ação poderá atingir ao final do pregão apenas indica a tendência.

\section{USO DE ENSEMBLES DE REDES NEURAIS}
Outro trabalho bastante interessante na previsão do mercado acionário é o de \citeonline[]{GIACOMELL} que diferentemente de Bosaipo, preferiu criar e treinar uma rede que apresente ao investidor apenas a tendência do mercado, informando se os próximos movimentos serão de alta ou queda. 
Em sua pesquisa Giacomel utiliza dois ensembles de redes neurais, sendo eles específicos para cada perfil de investidor, moderado ou agressivo. No ensemble moderado o autor escolheu usar como entrada os valores de abertura, máximo, mínimo, e fechamento dos dias anteriores, com essas informações o ensemble pode produzir três saídas, alta, queda ou não sabe que representa períodos de incertezas. Este ensemble é composto por duas redes neurais e mais uma porta lógica AND que faz a junção das saídas da rede.

O ensemble agressivo é um complemento do ensemble moderado, ele usa as mesmas entradas porém possui além das duas redes neurais um indicador técnico que avalia a situação do mercado e serve de apoio na decisão. O indicador técnico usado pelo autor foi o SAR pois segundo o mesmo é o que apresentou melhor desempenho quando comparado com outros.

O autor conseguiu alcançar resultados satisfatórios com o modelo proposto, porém os lucros poderiam ser mais altos se o investidor soubesse o valor máximo e mínimo que a ação poderia alcançar no dia, pois assim saberia o momento perfeito para comprar ou vender suas ações.

\section{CONSIDERAÇÕES}
Observando as pesquisas realizadas nesta área, foi constatado que os produtos existentes utilizam apenas uma série de retornos para prever as variações futuras, o que torna o resultado inconsistente e suscetível a erros, além do mais os artigos que foram pesquisados estão preocupados em apontar apenas a tendência de uma ação, ou seja, se esta irá subir ou descer no próximo período. O presente trabalho propõe a análise de vários fatores, como o valor de abertura, o valor de fechamento do dia anterior volume negociado, valor máximo e mínimo atingido pela ação em cada período, e terá como objetivo principal apontar o valor exato(ou o mais próximo possível) dos valores futuros para que o investidor saiba o melhor momento para fazer a compra ou venda de suas ações.
